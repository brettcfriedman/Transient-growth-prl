\documentclass[showpacs,preprintnumbers,amsmath,amssymb,superscriptaddress,aip]{revtex4-1}
\usepackage{graphicx}
%\setlength{\parindent}{0in}

% General Latex  --------------------------------------------------
\def\beq{\begin{equation}}
\def\eeq{\end{equation}}
\def\beqar{\begin{eqnarray}}
\def\eeqar{\end{eqnarray}}
\def\nn{\nonumber}
\def\ol{\overline}
\def\para{\parallel}

% Operators  ------------------------------------------------------
\newcommand{\diff}[2]{\frac{d#1}{d#2}}
\newcommand{\diffs}[2]{\frac{d^2#1}{d#2^2}}
\newcommand{\pdiff}[2]{\frac{\partial#1}{\partial#2}}
\newcommand{\pdiffs}[2]{\frac{\partial^2#1}{\partial#2^2}}
\newcommand{\pdiffxy}[3]{\frac{\partial^2#1}{\partial#2 \partial#3}}
\newcommand{\pdt}{\partial_t}
\newcommand{\pdr}{\partial_r}
\newcommand{\pdth}{\partial_\theta}
\newcommand{\pdrr}{\partial^2_r}

\newcommand{\enum}[2]{{#1}\times10^{#2}} % 4.2x10^{3} = \enum{4.2}{3}

\newcommand{\vect}[1]{{\bf #1}}
%\newcommand{\vect}{\overrightarrow}
%\newcommand{\vect}{\vec}
\def\div{\nabla\cdot}
\def\grad{\nabla}
\def\curl{\nabla\times}
\newcommand{\gradpar}{\grad_\parallel}
\newcommand{\gradperp}{\grad_\perp}
\newcommand{\gradr}{\grad_r}
\newcommand{\defeq}{\ensuremath{\stackrel{\text{\tiny def}}{=}}}

\newcommand{\savg}[1]{\left<{#1}\right>}
\newcommand{\vavg}[1]{\left<{#1}\right>_V}
\newcommand{\thavg}[1]{\left<{#1}\right>_\theta}

% Variable names  -------------------------------------------------
\newcommand{\vpar} {v_\parallel}
\newcommand{\Apar} {A_\parallel}
\newcommand{\jpar} {j_\parallel}
\newcommand{\kpar} {k_\parallel}
\newcommand{\kperp} {k_\perp }
\newcommand{\vperp} {v_\perp }
\newcommand{\kthe}{k_\theta}

\newcommand{\Evec}{\ensuremath{\boldsymbol{{\rm E}}}}
\newcommand{\Bvec}{\ensuremath{\boldsymbol{{\rm B}}}}
\newcommand{\Jvec}{\ensuremath{\boldsymbol{{\rm J}}}}
\newcommand{\Fvec}{\ensuremath{\boldsymbol{{\rm F}}}}
\newcommand{\fvec}{\ensuremath{\boldsymbol{{\rm f}}}}
\newcommand{\vE}{\ensuremath{\boldsymbol{{\rm v}_{E}}}}
\newcommand{\bo}{\ensuremath{\boldsymbol{{\rm b}_0}}}
\newcommand{\bvec}{\ensuremath{\boldsymbol{{\rm b}}}}
\newcommand{\xvec}{\ensuremath{\boldsymbol{{\rm x}}}}
\newcommand{\yvec}{\ensuremath{\boldsymbol{{\rm y}}}}
\newcommand{\zvec}{\ensuremath{\boldsymbol{{\rm z}}}}
\newcommand{\vvec}{\ensuremath{\boldsymbol{{\rm v}}}}
\newcommand{\jvec}{\ensuremath{\boldsymbol{{\rm j}}}}

\newcommand{\bxgp}{\bvec\times\gradperp}

\newcommand{\vve}{\ensuremath{\boldsymbol{{\rm v}}_{e}}}
\newcommand{\vvi}{\ensuremath{\boldsymbol{{\rm v}}_{i}}}
\newcommand{\vpe}{v_{\parallel e}}
\newcommand{\vpi}{v_{\parallel i}}
\newcommand{\vvE}{\ensuremath{\boldsymbol{{\rm v}}_{E}}}
\newcommand{\vvD}{\ensuremath{\boldsymbol{{\rm v}}_{D}}}

\newcommand{\nuei}{\nu_{ei}}
\newcommand{\nuii}{\nu_{ii}}
\newcommand{\nue}{\nu_{e}}
\newcommand{\nuen}{\nu_{en}}
\newcommand{\nuin}{\nu_{in}}
\newcommand{\kpe}{\kappa_{\parallel e}}

\newcommand{\rs}{\rho_{s}}
\newcommand{\ri}{\rho_{i}}
\newcommand{\wci}{\Omega_{i}}
\newcommand{\wcix}{\Omega_{ix}}
\newcommand{\wce}{\Omega_{e}}
\newcommand{\tomega}{\tilde\omega}
\newcommand{\Isat}{I_{\rm sat}}
\newcommand{\fmie}{\frac{m_i}{m_e}}
\newcommand{\fmei}{\frac{m_e}{m_i}}


% Often used dimensions
\newcommand{\cm}{\rm cm}
\newcommand{\mm}{\rm mm}
\newcommand{\cmn}{{\rm cm}^{-3}}
\newcommand{\mn}{{\rm m}^{-3}}
\newcommand{\eV}{\rm eV}
\newcommand{\G}{\rm G}
\newcommand{\T}{\rm T}



\begin{document}

\title{Non-modal Growth in LAPD Turbulence}

\author{B. Friedman}
\email{friedman@physics.ucla.edu}

\author{T.A. Carter}

\affiliation{Department of Physics and Astronomy, University of California, Los Angeles, California 90095-1547, USA}



\begin{abstract}
Large Plasma Device (LAPD) [W. Gekelman \emph{et al.}, Rev. Sci. Inst. {\bf 62}, 2875 (1991)] 
\end{abstract}

\maketitle

\section{Introduction}

\section{The Simulation Model}
\label{dw_model}

A Braginskii-based fluid model~\cite{Braginskii1965} is used to simulate global drift wave turbulence in LAPD using the BOUT++ code~\cite{dudson2009}. 
The evolved variables in the model are the plasma density, $N$, the electron fluid parallel velocity $\vpe$, the potential vorticity $\varpi \equiv \gradperp \cdot (N_0 \gradperp \phi)$,
and the electron temperature $T_e$. The ions are assumed cold in the
model ($T_i = 0$), and sound wave effects are neglected ($v_i = 0$). Details of the simulation code, derivations of the model, grid convergence studies, and analyses of simplified models
may be found in previously published LAPD simulation studies~\cite{Popovich2010a,Popovich2010b,Umansky2011,friedman2012,friedman2012b}.

The equations are developed with Bohm normalizations: lengths are
normalized to the ion sound gyroradius, times to the ion
cyclotron time, velocities to the sound speed, densities to the equilibrium peak density, and electron
temperatures and potentials to the equilibrium peak electron temperature. These normalizations are constants (not functions of radius) and are calculated from these reference values:
the magnetic field is $1$ kG, the ion unit mass is $4$, the peak density is $2.86 \times 10^{12}$ cm$^{-3}$, and the peak electron temperature
is $6$ eV. The equations are:

\beqar
\label{ni_eq}
\pdt N = - {\mathbf v_E} \cdot \grad N_0 - N_0 \gradpar \vpe + \mu_N \gradperp^2 N + S_N + \{\phi,N\}, \\
\label{ve_eq}
\pdt \vpe = - \fmie \frac{T_{e0}}{N_0} \gradpar N - 1.71 \fmie \gradpar T_e  + \fmie \gradpar \phi - \nue \vpe + \{\phi,\vpe \}, \\
\label{rho_eq}
\pdt \varpi = - N_0 \gradpar \vpe  - \nuin \varpi + \mu_\phi \gradperp^2 \varpi + \{\phi,\varpi \}, \\
\label{te_eq}
\pdt T_e = - {\mathbf v_E} \cdot \grad T_{e0} - 1.71 \frac{2}{3} T_{e0} \gradpar \vpe + \frac{2}{3 N_0} \kpe \gradpar^2 T_e  \nonumber \\
- \frac{2 m_e}{m_i} \nue T_e  + \mu_T \gradperp^2 T_e +  S_T + \{\phi,T_e\}.
\eeqar

In these equations, $\mu_N$, $\mu_T$, and $\mu_\phi$ are artificial diffusion and viscosity coefficients used for subgrid dissipation. They are large enough to allow saturation
and grid convergence~\cite{friedman2012}, but small enough to allow for turbulence to develop. In the simulations, they are all given the same value of $1.25 \times 10^{-3}$ in Bohm-normalized units. 
This is the only free parameter in the simulations. All other parameters such as the electron collisionality $\nue$, ion-neutral
collisionality $\nuin$, parallel electron thermal conductivity $\kpe$, and mass ratio $\fmie$ are calculated from experimental quantities.
There are two sources of free energy: the density gradient due to the equilibrium density profile $N_0$, and the equilibrium electron temperature gradient in $T_{e0}$, both of which are
taken from experimental fits. $N_0$ and $T_{e0}$ are functions of only the radial cylindrical coordinate $r$, and they are shown in Fig.~\ref{eq_profiles}. 
The mean potential profile $\phi_0$ is set to zero in the model, and terms involving $\phi_0$ are not included in Eqs.~\ref{ni_eq}-\ref{te_eq}. 
The justification for this is that biasable azimuthal limiters in LAPD allow for the mean $E \times B$ flow and flow shear to be varied with high precision, even allowing the flow to be
nulled out~\cite{schaffner2012}. 
The simulations in this paper use the $N_0$, $T_{e0}$, and $\phi_0$ profiles from the nulled out flow experiment, justifying setting the mean potential profile to zero in the simulations.

Simulations also use density and temperature sources ($S_n$ and $S_T$) in order to keep the equilibrium profiles from relaxing away from their experimental shapes. 
These sources suppress the azimuthal averages ($m=0$ component of the density and temperature fluctuations) at each time step. 
The azimuthal average of the potential $\phi$ is allowed to evolve in
the simulation, allowing zonal flows to form, although they are relatively unimportant to the turbulent dynamics~\cite{friedman2012b}.

The terms in Poisson brackets are the $E \times B$ advective nonlinearities, which are the only nonlinearities used in the simulations.
The numerical simulations use finite differences in all three dimensions and use cylindrical annular geometry ($12<r<40$ cm). The radial extent used in the simulation
encompasses the region where fluctuations are above a few percent in the experiment. Therefore, the radial boundaries are fixed to zero value.

\section{Conclusion}
\label{conclusion}


\begin{acknowledgments}

\end{acknowledgments}


%%%%%%%%%%%%%%%%%%%%%%%%%%%%%%%%%%%%%%%%%%%%%%%%%%%%%%%%%%%%%%%%%%%%%%%%%%%


\bibliography{refs}
%\bibliographystyle{unsrt}


\end{document}
